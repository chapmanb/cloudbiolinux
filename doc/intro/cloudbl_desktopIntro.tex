\section{The CloudBioLinux Desktop}
\label{section:cloudblDesktop}
\paragraph{}This section provides only a few tips about the CloudBioLinux desktop. As CloudBioLinux is based on Ubuntu, we recommend referring to online Ubuntu documentation for further details about the system in general. 

\subsection{Bioinformatics documentation}

\paragraph{}CloudBioLinux comes with a categorised bioinformatics documentation system, which provides information about the bioinformatics software on the system. This is the easiest place to find out about what bioinformatics programs are available, and where to find out more about how to use them. 

\paragraph{}At the moment, the easiest way to get to the installed documents is to open up a web browser (just click on the little firefox icon in the top taskbar), and then enter the address:  \textbf{file:///var/www/bioinformatics/index.html}. See figure \ref{fig:bldocs}.

\begin{figure}[!hd]
	\fbox
	{
		\begin{minipage}{13cm}
\includegraphics[width=\maxwidth]{"images/blbioinfDocumentation"}
\caption[Bioinformatics docs]{\label{fig:bldocs}A categorised bioinformatics documentation system. Perusing the information in these pages is probably the easiest way to start finding out what is available on your CloudBioLinux image.}
		\end{minipage}
	}
\end{figure}


\subsection{The bioinformatics menu}

\paragraph{}Some of the bioinformatics software on the system can be accessed via the graphical bioinformatics menu. See figure \ref{fig:bioinfmenu}. 

\begin{SCfigure}[][t]
\includegraphics[width=40mm]{"images/bioinfMenuNX2"}
\caption[Bioinformatics menu]{\label{fig:bioinfmenu}Some bioinformatics programs can be launched from the Bioinformatics menu. For a more comprehensive listing of the bioinformatics software on the system, check out the installed bioinformatics documentation pages. See figure \ref{fig:bldocs}.}
\end{SCfigure}


\subsection{Opening a terminal}

\paragraph{}Much of the bioinformatics software on CloudBioLinux systems can only be launched from the command line. This includes command line tools, and also some graphical software. In addition, much of the power of Linux comes through using the command line. You can easily open a terminal, in which you can type commands, when logged into a full desktop session. Just go to the \textbf{Accessories submenu} under the Applications menu in the top taskbar, and choose \textbf{Terminal}. See figure \ref{fig:NXterminal}. 


\begin{SCfigure}[][t]
\includegraphics[width=40mm]{"images/openTerminalNX"}
\caption[Open terminal]{\label{fig:NXterminal}You can open a command line terminal from under the Accessories submenu of the Applications menu in the top taskbar.}
\end{SCfigure}
